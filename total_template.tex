\documentclass[11pt]{article}
\usepackage[utf8]{inputenc}
%----------FONT AND GEOMETRY DEFINITION----------%
\usepackage[a4paper, hmargin=3.0cm, vmargin=3.0cm]{geometry}
\usepackage{helvet}
\renewcommand{\familydefault}{\sfdefault} %setting arial font (fake arial, no one will notice)
\usepackage{setspace}
\onehalfspacing %\setstretch{1.5} fa el mateix, però es pot canviar el decimal al que es vulgui
\usepackage[english]{babel}
\usepackage{fullpage}
\usepackage{afterpage}
\usepackage{lastpage}
\usepackage{fancyhdr}
\usepackage{graphicx}
\usepackage{xcolor}
\usepackage{hyperref}
\hypersetup{%
  colorlinks=false,
  linkcolor=black,
  linkbordercolor={1.0 1.0 1.0},
  breaklinks = true,
  citebordercolor={1.0 1.0 1.0},
  hidelinks = true,
}
\usepackage{xurl}
\usepackage{array}
%\usepackage{wrapfig}
%\usepackage{multirow}
%\usepackage{tabularx}
%\usepackage{multicol}
\usepackage{caption}
%\usepackage{subcaption}
%\usepackage{float}

%----------BIBLIOGRAPHY----------%
%cal tenir un fitxer d'extensió .bib (els fa Mendeley) anomenat biblio a la carpeta on es compili
\usepackage[backend=biber, style = chem-acs]{biblatex}
\usepackage{csquotes}
%en cas d'escriure la tesi en català o catellà:
%\usepackage[autostyle=false, style=spanish]{csquotes}

% es pot mirar els tipus de documents en el fitxer .bib que biblatex podrà editar a: https://mirror.las.iastate.edu/tex-archive/macros/latex/contrib/biblatex-contrib/biblatex-chem/chem-acs.bbx

%---màxim de noms a la bibliografia
\ExecuteBibliographyOptions{maxnames=10}
%----
%-----Per fer que l'issue en articles surti entre parèntesi després del volum
\DeclareFieldFormat[article]{issue}{\mkbibparens{#1}}
\renewbibmacro*{volume+number+eid}{%
  \printfield{volume}%
  \setunit{\space}%
  \printfield{issue}%
  \setunit{\addcomma\space}%
  \printfield{eid}}
%---------------------------------------------------------
\addbibresource{biblio.bib}

%----------TITLE AND PAGE FORMATTING----------%
\newcommand\NomComplet{Manel Bayon Jimenez}
\newcommand\NIU{9999999}
\newcommand\NomProfeA{Manel Bayon Jimenez}
\newcommand\NomProfeB{Manel Bayon Jimenez}
\newcommand\CodiTFG{TFG2122\_xxx}
\newcommand\TitolTFG{El meu TFG té tal títol}
\newcommand\NomCentre{Departament de Química } %cal posar un espai al final
\newcommand\NomGrau{Química}
\newcommand\Mes{Juny } %cal espai al final
\newcommand\Any{2022}

\fancypagestyle{sumari}{
\fancyhf{}
\rhead{\TitolTFG}
\renewcommand{\headrulewidth}{0pt}%
\renewcommand{\footrulewidth}{0pt}%
}

\fancypagestyle{afteres}{
\fancyhf{}
\headheight 0 cm
\headsep 1.725cm
\rhead{\TitolTFG}
\rfoot{\thepage}
\renewcommand{\headrulewidth}{0pt}%
\renewcommand{\footrulewidth}{0pt}%
}

\pagestyle{fancy}
\fancyhf{}
\headheight 0 cm
\headsep 1.725cm
\rhead{\TitolTFG}
\rfoot{\thepage}
\renewcommand{\headrulewidth}{0pt}%
\renewcommand{\footrulewidth}{0pt}%

%----------ABBREVIATIONS LIST----------%
\usepackage{glossaries}
\usepackage{glossary-longragged}

\makenoidxglossaries

\newacronym{bams}{BAMS}{bar-assisted meniscus shearing}
\newacronym{ofet}{OFET}{organic field-effect transistor}
\newacronym{osc}{OSC}{organic semiconductor}
\newacronym{sam}{SAM}{self-assembled monolayer}

\glsaddall[types=\acronymtype] % to add all the acronyms, cited or not in the text


%----------DOCUMENT COMPILATION----------%
\begin{document}
\newgeometry{bottom = 1cm}
\begin{titlepage}

\center % Center everything on the page

%	LOGO SECTION
%----------------------------------------------------------------------------------------

\includegraphics[width = 6.1cm]{nom_uab.jpg}\\[1.76cm] % Include a department/university logo - this will require the graphicx package

% TITLE SECTION
%----------------------------------------------------------------------------------------

\textbf{\LARGE Facultat de Ciències}\\[3.53cm] % Major heading such as course name
%\rule{.1pt}{5cm}
\begin{flushright}
\begin{tabular}{r | p{.5\textwidth}}
  {\Large Treball de} & {\LARGE \CodiTFG}\\
  {\LARGE fi de grau}  & {\LARGE \TitolTFG}
\end{tabular}
\end{flushright}\vspace{7.06cm}


%----------------------------------------------------------------------------------------
%	AUTHOR SECTION
%----------------------------------------------------------------------------------------


\begin{tabular}{p{.48\textwidth} p{.48\textwidth}}
\large
Direcció: & Alumne: \\
\NomProfeA & \NomComplet\\
\NomProfeB & NIU: \\
 & \NIU
\end{tabular}
\vspace{0.7cm}


%----------------------------------------------------------------------------------------
%	DATE SECTION
%----------------------------------------------------------------------------------------

{\large \Mes \Any}\\[1.4cm] % Date, change the \today to a set date if you want to be precise

%----------------------------------------------------------------------------------------
Treball de fi de grau realitzat al \NomCentre i presentat a la\\ Facultat de Ciències\\ de la Universitat Autònoma de Barcelona per a l'obtenció del Grau en \NomGrau

%\vfill % Fill the rest of the page with whitespace

\end{titlepage}
\restoregeometry
\pagenumbering{roman}
%\afterpage{\aftergroup\restoregeometry}
\thispagestyle{empty}
{\centering \textcolor{white}{This page is left blank intentionally}}
\setcounter{page}{0}
\newpage

\thispagestyle{afteres}
\section*{Resum analític}

blabla

\newpage

\section*{Resumen Analítico}

blebleble

\newpage

\section*{Analytical abstract}

bliblibli

\newpage
\thispagestyle{sumari}
%\rhead{\TitolTFG}
\tableofcontents{}

\newpage
\setcounter{page}{1}
\pagenumbering{arabic}
\rhead{\TitolTFG}
\section{List of abbreviations}
\renewcommand{\glsnamefont}[1]{\textbf{#1}}
\printnoidxglossary[type=main, title={\vspace{-1cm}}, nonumberlist, nogroupskip, style=super]%nonumberlist deletes the list of links in the index to every acronym listed. nogroupskip deletes the extra space between letter groups
\newpage

\section{Introduction}
The technique \gls{bams} will be used in this thesis. The \gls{bams} is commonly known for...
\newpage

\section{Objectives}

\newpage

\section{Methodology}
\Gls{bams}
\newpage

\section{Results and Discussion}

\newpage

\section{Conclusions}
If we get some ideas from a paper but we're not citing it explicitly we can use the command $\backslash$nocite and it will be added in the bibliography in the desired order\nocite{Lamport2018}. %article example

Another example of an article citation is provided here\autocite{Foster2019} . %article example with explicit citation in the document body

Check the .bib file to see how a PhD thesis is properly formatted. As an example, Temiños thesis is cited here\autocite{TeminoGutierrez2019} .

A review cited is also a good example of the documents that might need to be cited during a thesis\autocite{Bronstein2020} .
\newpage

\section{Bibliography}


\printbibliography[title = { \vspace{-1cm}}]
\end{document}
